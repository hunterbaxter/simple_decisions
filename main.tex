\documentclass{beamer}
\usepackage[utf8]{inputenc}

\title{Simple Decisions}
\author{Hunter Baxter}

% themes found here: https://latex-beamer.com/tutorials/beamer-themes/
\usetheme{PaloAlto}
\setbeamertemplate{footline}[frame number]

\begin{document}
\maketitle

\begin{frame}
\frametitle{Presentation Roadmap}
\begin{itemize}
    \item Preferences
    \item Utility Theory 
    \item Decision Networks
    \item Value of Information
\end{itemize}
\end{frame}

\section{Rationality}
\begin{frame}{Rationality}
\end{frame}

\section{Preferences}
\begin{frame}{Preferences}
We require the ability to compare the degree of desirability of outcomes.

\subsection{Preference Operators}
Preference Operators:
\begin{itemize}
    \item $A \succ B$ if one prefers A over B
    \item $A \sim B$ if one is indifferent between A and B
    \item $A \succeq B$ if one prefers A over B or is indifferent
\end{itemize}
\end{frame}

\begin{frame}{Lottery}
Preference operators can be used to compare preferences over uncertain outcomes, not just fixed events. \\
A \textit{lottery} is a set of probabilities associated with a set of outcomes.

Let $S_{1:n}$ be a set of outcomes with associated probabilities $p_{1:n}$. \\

The lottery involving $S$ would be denoted as follows:
$[S_1:p_1;...;S_n;:p_n]$

\end{frame}

\begin{frame}{Rational Preferences}
Rational Preferences:
\begin{itemize}
    \item \textit{Completeness.} Exactly one of the following holds: $A \succ B, A \sim B, A \prec B$
    \item \textit{Transitivity.} $(A \succeq B) \land (B \succeq C) \implies A \succeq C$
    \item \textit{Continuity.} $A \succeq B \succeq C \implies \exists$ probability $p$ such that $[A: p; B: 1-p] \sim C$
    \item \textit{Independence.} $A \succ B \implies \forall C$ and $\forall$ probability $p, [A:p;C:1-p] \succeq [B:p;C:1-p]$
\end{itemize}
TODO: Mention that humans are not rational, but we are only going to focus on rational decisions
\end{frame}


\section{Utility Theory}
\begin{frame}{Utility Theory}
\end{frame}

\section{Decision Networks}
\begin{frame}{Decision Networks}
\end{frame}

\section{Value of Information}
\begin{frame}{Value of Information}
\end{frame}

\begin{frame}{Conclusion}
\end{frame}

\end{document}
